%\subsection{Proof}
%Let us explicitly calculate the derivative on the LHS of
%\eq{A2.2:kelvin}.
%\begin{equation}
%D_t \int_S \oo\cdot\nhat dS = \int_S \frac{\partial\oo}{\partial t}\nhat dS +  
%\end{equation}

%\begin{subappendices}
%\section{Dynamical systems, existence, uniqueness and all that}
%In two dimensions the differential equation that the streamlines
%satisfies is simply
%\begin{equation}
%\frac{dx}{F(x,y)}=\frac{dy}{G(x,y)}
%\end{equation}
%There are some cases where these can be simply solved -- what I
%describe below is borrowed from the book by Bender and Orszag~\cite{Bender2013advanced}.
%\subsection{Separable equations }
%When the function $F$ and $G$ are such that 
%\begin{equation}
%\frac{G}{F} =a(x)b(y)\/. 
%\end{equation}
%In this case we end up with the following equation:
%\begin{equation}
%\frac{dy}{dx} = a(x)b(y) \/,
%\end{equation}
%an equation that can be readily integrated. 
%\subsection{Exact Equations}
%The equation for streamlines can in general be written as 
%\begin{equation}
%\frac{dx}{F(x,y)} = \frac{dy}{G(x,y)}\/\quad ,
%\end{equation}
%from which we can write
%\begin{equation}
%M(x,y) + N(x,y)\frac{dy}{dx} = 0 
%\label{A2.2:exact}
%\end{equation}
where $N = 1/G$ and $M=-1/F$.
If $M$ and $N$ satisfies 
\begin{equation}
\frac{\partial}{\partial y} M(x,y) = \frac{\partial}{\partial x}
N(x,y) \/,
\end{equation}
 then the equations are called exact and can write the LHS of
 \eq{A2.2:exact} as 
\begin{equation}
\frac{d}{dx} f(x,y) = 0
\end{equation}
This is a first order differential equation in two variables than is
straightforward to integrate. 
\subsection{Existence and Uniqueness}
\begin{thm-non}
For a differential equation of the form 
\begin{equation}
\frac{dy}{dx} = f(x,y)
\end{equation}
let $f(x,y)$ be continious on a rectangle 
$D = \{(x,y): 
\xnot-\delta <x < \xnot +\delta ;
 \ynot-b <y < \ynot +b\} $then there exists a solution 
in $D$. Furthermore, if $f(x,y)$ is \textit{Lipschitz continious} with
respect to $y$ on a rectanngle $R$ (possibly smaller than $D$) given
as
$D = \{(x,y): 
\xnot-a <x < \xnot +a ;
 \ynot-b <y < \ynot +b, a<\delta \} $ the solution in $R$ is unique. 
\end{thm-non}
I am not trying to give a proof here, but let us try to understand the
statement. The first thing I need to define is Lipschitz continuity. 
\subsection{Lipschitz continuity}
\begin{def-non}
Any function $f(y)$ is called (locally) Lipschitz continious if 
\begin{equation}
\mid f(y+\delta y) - f(y) \mid < L (\delta y) \quad
\text{as}\quad \delta y \to 0
\end{equation}
\end{def-non}
Simply stated the function is Taylor expandable which in turn means
that $dy/dx$ exists at $x=xnot$. To see this clearly, it is useful to
consider an example. 
\begin{equation}
\frac{dy}{dx} = \frac{1}{x^{1/3}}
\end{equation}
The solution is 
\begin{equation}
y(x) = x^{2/3} \text{and} y = 0
\end{equation}
%\subsection{What is integrability?}
%\end{subappendices}