
\documentclass[a4paper,twoside,10pt]{article}

\usepackage{amsfonts,enumerate,bm,color,graphicx}
\input{symdef.inc}
\setlength{\textheight}{24cm}
\setlength{\textwidth}{14cm}
\setlength{\topmargin}{-1.5cm}
\setlength{\oddsidemargin}{0.5cm}
\setlength{\evensidemargin}{-0.5cm}
\pagestyle{empty}


\begin{document}

%--- CUSTOMIZATION COMMANDS ---%

\newenvironment{itmz}%
{ \vspace{-\parskip} \begin{itemize} }%
{ \end{itemize} }
\def\flux{{\cal F}_\nu}

%--- TITLE ---%

\begin{center}
  {\Large \bf
   Astrophysical MHD ( AS7019)\\[2mm]
   Homework II} \\[4mm]
To be returned on Thursday 15th of February 
\end{center}
All the problem carry 25 marks.  
\bigskip

%--- PROBLEMS ---%



\begin {enumerate}

\item Solution of Stokes flow around a sphere. 
%-----------------------------------
\begin{figure}[h]
\includegraphics[width=0.6\columnwidth]{fig/StokesProblem.png}
\caption{\small{A solid sphere in a flow. At large distance from the sphere
  the flow is give by a constant velocity $\VV$. The radius of the
  sphere is unity. The vector $\rr$ is the vector to any point in
  space with the center of the sphere as origin. }
}\label{fig:st}
\end{figure}
%-----------------------------------
Continuing from where we left off in class show that 
\begin{equation}
\chi(r) = \frac{1}{4}r^2 + Ar + \frac{B}{r}
\end{equation}
is a solution of the biharmonic equation
\begin{equation}
\biharmonic \chi = 0 
\end{equation}
where $r$ is the radian coordinate in spherical polar coordinate
system. $A$ and $B$ must be chosen such that 
\begin{equation}
u_{\alpha} = U_{\alpha\beta}u_{\beta} 
\end{equation}
with 
\begin{equation}
U_{\alpha\beta} = \delta_{\alpha\beta}\lap \chi - \frac{\partial^2 \chi}{\partial
  x_{\alpha}\partial x_{\beta}}
\end{equation}
must be zero at $r=1$. This implies imposition of no-slip boundary
conditions as all components of velocity goes to zero at the boundary
of the sphere whose radius is assumed to be unity. 
Show that this boundary condition implies 
\begin{equation}
\chi^{\prime\prime}(1) = \chi^{\prime}(1) = 0  
\end{equation}
From this show that 
$B={\textstyle{1\over 4}}$ and
$A={\textstyle{3\over 4}}$.
The show that for $r > 1$ 
\begin{equation}
u_{\alpha} = V_{\alpha} - \frac{3}{4}\left(
  \frac{V_{\beta}r_{\beta}r_{\alpha}}{r^3} + \frac{V_{\alpha}}{r}
    \right)
-\frac{1}{4}\frac{\partial}{\partial x_{\alpha}}\left( \frac{V_{\beta}r_{\beta}}{r^3}\right)
\end{equation}

\item Write down the equations of isothermal MHD and then
  non-dimensionalize the equations. Assume that there is a
  characteristic length scale $\ell = \frac{1}{\kf}$, velocity scale
  $u$, constant sound-speed $\cs$, a magnetic field of magnitude
  $B_{0}$, and a constant background density $\rho_{0}$. Show that the
  non-dimensionalized equations have the following dimensionless
  parameters:
\begin{itemize}
\item Reynolds number : $\Rey = \frac{u}{\nu\kf}$, 
\item magnetic Reynolds number: $ \Rm = \frac{u}{\eta\kf}$, 
\item Mach number: $\Ma =\frac{u}{\cs}$, 
\item Alfvenic Mach number: $\MaA = \frac{u}{\cA}$,
\end{itemize}
where $\cA = \frac{B_{0}}{\rho_{0}\mu_{0}}$.

\item Work out the solution of the solar wind problem assuming the
  flow is isentropic (adiabatic) instead of the isothermal assumption
  made in class. 
\item Functional methods: Among all the curves joining two given
  points $(x_{0},y_{0})$ and $(x_{1},y_{1})$, find the one which
  generates the surface of minimum area when rotated about the $x$-axis. 
\end{enumerate}
\end{document}

